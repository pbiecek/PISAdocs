\section{Rankings}

First we will see how to create ranking for countries.
It's easy to do this for all countries, but since there is a lot of them in order to save space we will do this for european countries only.

So first, let's subselect only european countries. The dataset PISAeurope will have the sale variables like student2009 but only rows for students from Europe. \footnote{In remaining examples you can replace PISAeurope by student2009 and then you will get ranking for all countries.}
\begin{shaded}\begin{verbatim}
europe <- getCountryIdsFromRegion(region="Europe")
PISAeurope <- student2009[student2009$CNT %in% europe, ]
\end{verbatim}\end{shaded}

First let's calculate weighted average performance for all these countries. You can do this with the function \verb:getWeightedAverages():. You need to specify three arguments. Variables with performance (here Plausible values form Math, Reading and Science), grouping variable (here country) and weights.

As a result you will get data.frame with weighted averages.

\begin{shaded}\begin{verbatim}
getWeightedAverages(PISAeurope[,c("PV1MATH", "PV1READ", "PV1SCIE")], 
       factor(PISAeurope$CNT), PISAeurope$W_FSTUWT)
       
     PV1MATH  PV1READ  PV1SCIE
ALB 376.8412 384.8553 390.0552
AUT 495.3798 470.0181 494.3429
BEL 515.6946 506.0709 506.9133
BGR 427.8899 428.7424 439.2365
CHE 535.0264 500.1675 517.0095
CZE 492.5683 478.3270 500.8206
DEU 512.0990 497.2816 520.2056
DNK 503.2260 495.2494 498.9904
ESP 483.7105 480.9529 488.4244
EST 512.0336 500.3432 527.5742
FIN 540.4180 535.5694 553.6443
FRA 496.7549 495.3661 497.8581
GBR 492.5210 493.9524 513.6889
GRC 465.4492 481.8291 469.3762
HRV 460.6449 475.8268 486.8437
HUN 489.9584 494.2865 502.2476
IRL 487.3271 495.9401 508.2677
ISL 507.3673 500.5733 495.6023
ITA 483.2503 486.3324 489.2809
LIE 533.6883 498.4257 518.5018
LTU 476.4921 467.9608 490.9942
LUX 488.1766 471.1547 483.1795
LVA 481.4847 483.5472 492.8094
MDA 397.3944 388.2060 413.1806
MLT 462.5997 442.1508 461.3761
MNE 402.7139 407.0327 401.5027
NLD 525.8939 508.1992 522.6339
NOR 497.5454 503.0985 499.1366
POL 494.2307 500.1981 507.4326
PRT 487.2701 489.1076 492.8555
ROU 426.4127 424.4139 428.0784
RUS 467.9225 459.4349 478.5926
SRB 442.6190 442.3280 442.8573
SVK 496.7076 477.4750 490.9146
SVN 501.0395 482.7662 511.2549
SWE 493.8699 497.7079 494.8899
\end{verbatim}\end{shaded}

But initially we were going to derive rankings. It's straight forward. Just use \verb:getRanking(): with same options like for \verb:getWeightedAverages():.

\begin{shaded}\begin{verbatim}
getRanking(PISAeurope[,c("PV1MATH", "PV1READ", "PV1SCIE")], 
                  factor(PISAeurope$CNT), PISAeurope$W_FSTUWT)
                  
    PV1MATH PV1READ PV1SCIE
ALB      36      36      36
AUT      14      27      19
BEL       5       3      11
BGR      32      32      32
CHE       2       8       6
CZE      17      23      13
DEU       6      11       4
DNK       9      14      15
ESP      23      22      25
EST       7       6       2
FIN       1       1       1
FRA      12      13      16
GBR      18      16       7
GRC      28      21      29
HRV      30      25      26
HUN      19      15      12
IRL      21      12       9
ISL       8       5      17
ITA      24      18      24
LIE       3       9       5
LTU      26      28      22
LUX      20      26      27
LVA      25      19      21
MDA      35      35      34
MLT      29      31      30
MNE      34      34      35
NLD       4       2       3
NOR      11       4      14
POL      15       7      10
PRT      22      17      20
ROU      33      33      33
RUS      27      29      28
SRB      31      30      31
SVK      13      24      23
SVN      10      20       8
SWE      16      10      18
\end{verbatim}\end{shaded}

Note that there is an optional argument \verb:sort:. You can select the column according to which rows should be sorted (by default they are sorted alphabetically).

\begin{shaded}\begin{verbatim}
getRanking(PISAeurope[,c("PV1MATH", "PV1READ", "PV1SCIE")], 
               factor(PISAeurope$CNT), PISAeurope$W_FSTUWT, sort=1)
               
    PV1MATH PV1READ PV1SCIE
FIN       1       1       1
CHE       2       8       6
LIE       3       9       5
NLD       4       2       3
BEL       5       3      11
DEU       6      11       4
EST       7       6       2
ISL       8       5      17
DNK       9      14      15
SVN      10      20       8
NOR      11       4      14
FRA      12      13      16
SVK      13      24      23
AUT      14      27      19
POL      15       7      10
SWE      16      10      18
CZE      17      23      13
GBR      18      16       7
HUN      19      15      12
LUX      20      26      27
IRL      21      12       9
PRT      22      17      20
ESP      23      22      25
ITA      24      18      24
LVA      25      19      21
LTU      26      28      22
RUS      27      29      28
GRC      28      21      29
MLT      29      31      30
HRV      30      25      26
SRB      31      30      31
BGR      32      32      32
ROU      33      33      33
MNE      34      34      35
MDA      35      35      34
ALB      36      36      36
\end{verbatim}\end{shaded}


\section{Use case - gender differences}

Here the distribution of performance between countries and genders can be presented



\section{Different approaches to ranking calculation}

\begin{itemize}
\item Standard PISA approach via PV
\item Percentage of correct answers
\item Percentage of correct answers after removing k outliers in items
\item Top 90\% percentile of the distribution
\item Absolute dominance defined as one country over performs other only it also over perform on more than 50\% of items
\end{itemize}








Here there should be an information how to do some simple statistics with the data.

Like weighted regression.

Maybe mixed effect model or generalized mixed effect model.


\section{Exercises}

A set of exercises to practice working with data
With answers on the end of book.

